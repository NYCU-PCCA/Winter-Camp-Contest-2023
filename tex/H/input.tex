The input consists of five parts.
The first part contains one line, and that line 
contains two positive integers $n$ and $m$. 
$n$ is the number of nodes, and $m$ is the number of edges.
The second part contains $m$ lines.
Each of them contains two integers $u$ and $v$, indicating an edge $\{u, v\}$
of the given graph.
The third part contains one line.
That line consists of $n$ space-separated integers $x_1,x_2,\dots,x_n$. 
For any $k\in\{1,2,\dots,n\}$, if the node value $V_k$ is missing, 
$x_k$ will be \verb+-1+; otherwise, $V_k$ is $x_k$.
The fourth part contains one integer $q$, indicating the number of constraints.
The fifth part contains $q$ lines, and each of them contains
five space-separated integers $t, u, i, v, j$ indicating that 
$(t, u, i, v, j)$ is a constraint.
