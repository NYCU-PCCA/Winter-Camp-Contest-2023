Ian has a multiset $S$ of intervals on the number axis, $i$-th of which is $[l_i,r_i]$.
Surprisingly, he found that every interval is within the $[0,l]$ range, where $l$ is Ian's favorite positive integer.

As Ian thinks everything should be perfectly balanced, he wants to add the minimum number of intervals to $S$, so that the same number of intervals covers every non-integer coordinate between $0$ and $l$.
The newly added intervals should also be within the $[0,l]$ range.
Let $f(S)$ be the minimum number of intervals to be added.
Note that $S$ is not changed after calculating $f(S)$.

For example, consider the case that $S$ contains intervals $[0,3], [2,8], [7,10]$ and $l=10$.
Then Ian can add $3$ intervals $[0,2],[3,7],[8,10]$ to $S$, and then every non-integer coordinate between $0$ and $10$ are covered by $2$ intervals.
Therefore $f(S) = 3$ in this case.

Due to the instability of $S$, Ian observes that $S$ is prone to change.
He wonders if the value of $f(S)$ can be changed.
More formally, there are $q$ queries of three types:

\begin{enumerate}
\item $1\ ql_i\ qr_i$ — Adding an interval $[ql_i,qr_i]$ to $S$.
\item $2\ ql_i\ qr_i$ — Removing an interval $[ql_i,qr_i]$ to $S$. It is guaranteed that $[ql_i,qr_i]$ appears in $S$.
\item $3$ — Ian wants to know the value of $f(S)$.
\end{enumerate}

Can you help Ian answer all the queries?