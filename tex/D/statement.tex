Neko-chan and Laffey are a famous duo of magicians in the PCCA kingdom.
Recently they came up with the following new trick for their next show.

In the beginning, Neko-chan will invite an audience to write down a random permutation $a=[a_1,a_2,\ldots,a_n]$ of $1,2,\ldots,n$ without showing it to Laffey.
After seeing the permutation, Neko-chan will hand out a deck of cards (at most $n$ cards, otherwise, it would be too apparent that they are cheating) to another audience.
Neko-chan will ask the audience to shuffle the cards in any order and hand it to Laffey.
Here comes the miracle: Laffey then can read out a sequence of operations to swap elements in $a$, and the array $a$ would be sorted in increasing order!

If you wonder how this trick works, here is the secret.
After the array $a$ is written, Neko-chan secretly writes down a pair of integers $(x_i,y_i)$ on each card.
The pair $(x_i,y_i)$ denotes that Laffey have to swap $a_{x_i}$ and $a_{y_i}$.
When Laffey receives the deck of cards, she only has to sort the pairs on each card lexicographically  (that is, sort the pairs in $op$) and read out the sorted pairs.
In other words, Neko-chan gives Laffey an array of integer pairs $op$ (which is also a sequence of operations); all Laffey has to do is sort $op$.

To make the trick work, Neko-chan requires an efficient algorithm to help him produce the array of pairs $op$.
This task is too complex for Neko-chan. Can you help him?