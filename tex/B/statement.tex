A divisor of a positive integer $n$ is an integer $d$ where $m=\frac{n}{d}$ 
is an integer. 
In this problem, we define the aliquot sum $s(n)$ of a positive integer $n$ 
as the sum of all divisors of $n$ other than $n$ itself. 
For examples, $s(12)=1+2+3+4+6=16$, $s(21)=1+3+7=11$, and $s(28)=1+2+4+7+14=28$.

With the aliquot sum, we can classify positive integers into three types: 
abundant numbers, deficient numbers, and perfect numbers.
The rules are as follows.
\begin{enumerate}
\item A positive integer $x$ is an abundant number if $s(x)>x$.
\item A positive intewer $y$ is a deficient number if $s(y)<y$.
\item A positive integer $z$ is a perfect number if $s(z)=z$.
\end{enumerate}

You are given a list of positive integers. 
Please write a program to classify them.
