DNA is a polymer found in many organisms' cells, including humans.
It is composed of two strands, which coil together to form a double helix.

The structure of each DNA strand can be seen as an array of nucleotides.
According to the type of nucleobases each nucleotide contains, we can classify the nucleotides into four categories: \texttt{A}, \texttt{T}, \texttt{C}, and \texttt{G}.
Thus, each strand of the DNA can be simplified as a sequence containing characters \texttt{ATCG}.
The beginning and the end of any DNA strand are called the $5'$-end and the $3'$-end, respectively.

The nucleotides on the two strands bind together, which is why the two strands won't seperate.
According to base pairing rules, \texttt{A} is always paired with \texttt{T} while \texttt{C} is always paired with \texttt{G}.
The direction of the nucleotides in one strand is opposite to the direction in the other strand;
that is, one strand's $5'$-end matches the other strand's $3'$-end.
Hence, to find the sequence for the other strand, one just needs to reverse the sequence and do the substitution
(\texttt{A} $\to$ \texttt{T}, \texttt{T} $\to$ \texttt{A}, \texttt{C} $\to$ \texttt{G}, \texttt{G} $\to$ \texttt{C}).
See the figure below for a better understanding.

\begin{figure}[h]
\center
\includegraphics[width=0.5\textwidth]{image/dna.png}
\caption{The DNA structure.}
\end{figure}
    

Alan is a biologist conducting experiments with DNA.
He is playing with a special kind of DNA, just finding the sequence for one strand from the $5'$-end to the $3'$-end is $s$.
Can you help him find the sequence for the other strand from the $5'$-end to the $3'$-end? 